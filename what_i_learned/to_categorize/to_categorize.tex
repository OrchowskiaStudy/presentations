\chapter{Do uporządkowania}
Tu znajdują się rzeczy które nie znalazły jeszcze swojej kategorii i czekają na nią, oraz uporządkowanie.


\section{Architektura aplikacji}
To jest raczej moje przemyślenie odnośnie wytwarzania aplikacji ich struktury i działania.

Bardzo spodobało mi się podejście czy też filozofia UNIX. Ma ona kilka założeń które stoją u podstaw tego wspaniałego ekosystemu.
Co prawda nie można do końca tego tak spłycać, ale zauważyć można pewną prawidłowość odnoszącą się do wytwarzania systemów i ewolucji w stronę mikroserwisów.

	\begin{itemize}
		\item Każdy program powinien robić jedną rzecz. Lepiej zrobić kolejny program niż komplikować obecny.
		\item Spodziewaj się że output z jednego programu będzie inputem dla innego - nieznanego programu. Niech output będzie więc lakoniczny.
		\item Buduj oprogramowanie tak aby mogły być wypróbowane wcześnie, nawet jeśli nie wszystko jest jeszcze dopracowane. Zawsze można wyrzucić niepotrzebne fragmenty i je przebudować.
		\item Jeśli potrzebujesz narzędzi to je buduj, nawet jeśli ci nie po drodze. To ci pomoże przy programowaniu. Udostępnij je później, lub wyrzyć jeśli nie przynoszą wartości.(automatyzacja)
	\end{itemize}

Nie można odmówić tym systemom sukcesu, w końcu praktycznie każde urządzenie bazuje w tej chwili na LINUX, który również stosuje się do tej filozofii. 
Nie jest to powiedziane wprost, ale autonomia mikroserwisów wynika właśnie z tego - robią jedną rzecz dobrze (a przynajmniej powinny).
Popularny ostatnio model DevOps odnosi się do automatyzacji i wczesnego wydawania oprogramowania, a w złożonym systemie czasami nie wiemy nic o kliencie naszego API, więc również powinno być ono odpowiednio dostosowane.

Jako, że systemy te powstały już na początku lat 70, a mikroserwisy są stosunkowo bardzo młode - wiedza którą możemy wyciągnąć z tych doświadczeń może być bardzo pożyteczna, a do której tak na prawdę nie sięgamy - wolimy wynajdować koło na nowo, czytać o DDD, architektuuurze i innych tego typu rzeczach.

Przykładem może być IPC (ang. inter-process communication), która jest odpowiednikiem systemu zdarzeń w systemach służącego do wymiany danych pomiędzy nimi.

\subsubsection{PODSUMOWUJĄC}
TO JUŻ WSZYSTKO KURKA BYŁO!
