\chapter{Do uporządkowania}
Tu znajdują się rzeczy które nie znalazły jeszcze swojej kategorii i czekają na nią, oraz uporządkowanie.


\section{Architektura aplikacji}
To jest raczej moje przemyślenie odnośnie wytwarzania aplikacji ich struktury i działania.

Bardzo spodobało mi się podejście czy też filozofia UNIX. Ma ona kilka założeń które stoją u podstaw tego wspaniałego ekosystemu.
Co prawda nie można do końca tego tak spłycać, ale zauważyć można pewną prawidłowość odnoszącą się do wytwarzania systemów i ewolucji w stronę mikroserwisów.

	\begin{itemize}
		\item Każdy program powinien robić jedną rzecz. Lepiej zrobić kolejny program niż komplikować obecny.
		\item Spodziewaj się że output z jednego programu będzie inputem dla innego - nieznanego programu. Niech output będzie więc lakoniczny.
		\item Buduj oprogramowanie tak aby mogły być wypróbowane wcześnie, nawet jeśli nie wszystko jest jeszcze dopracowane. Zawsze można wyrzucić niepotrzebne fragmenty i je przebudować.
		\item Jeśli potrzebujesz narzędzi to je buduj, nawet jeśli ci nie po drodze. To ci pomoże przy programowaniu. Udostępnij je później, lub wyrzyć jeśli nie przynoszą wartości.(automatyzacja)
	\end{itemize}
